\documentclass[12pt]{article}
\usepackage[T1]{fontenc}
\usepackage[polish]{babel}
\usepackage[utf8]{inputenc}
\usepackage{lmodern}
\selectlanguage{polish}

\title{I ty możesz zostać magistrem}
\date{\today}


\begin{document}	
\maketitle
\tableofcontents

\section{Baza Bernsteina}
Baza Bernsteina ze względu na świetne właściwości numeryczne i geometryczne jest szeroko stosowana w systemach CAD/CAM pomimo faktu, że nie jest trójkątna (traingular ???) i nie jest najszybsza w obliczeniach.\\

Ważne właściwości Bazy Bernsteina:
\begin{itemize}
	\item Formują bazę w $n+1$ wymiarowej przestrzeni $w^{n}$ wszystkich wielomianów stopnia nie większego niż $n$.
	\item Sumują się do 1 dla każdego $t \in R$
	\item Są nienegatywne w przedziale $[0,1]$ i dodatnie w $(0,1)$.
	\item Sa symetryczne, tzn. $B^{n}_{i=0...n}(t) = B^{n}_{n-1}(1-t)$
\end{itemize}

\subsection{Algorytm de Casteljau}
Algorytm de Casteljau służy do obliczania wartości wielomianów w Bazie Bernsteina. Jest stabilny numerycznie. 
Niewielkim kosztem możemy uzyskąć nie tylko wartość, ale i pochodną w punkcie. Należy odczytać obie wartości algorytmu dla $n-1$ odjąć je i pomnożyć przez $n$.

\subsection{Twierdzenie Weierstrassa}
Każdą funkcję ciągłą o wartościach rzeczywistych na przedziale domkniętym $[a,b]$ można przybliżyć jednostajnie z dowolną dokładnością wielomianami.

\subsection{Baza Hermite'a}
Jeżeli znamy wartości na krańcach przedziału i znamy wartości pochodnych w tych punktach, to podstawiamy do wzoru i mamy aproksymację funkcji na przedziale.
\subsection{Baza Lagrange'a}
Baza nie jest triangularna (???). Można w niej interpolować.
\subsection{Węzły Czebyszewa}
Interpolacja w węzłach Czebyszewa jest prawie najlepsza (znika efekt Rungego).

\section{Piecewise polynomials}
\subsection{Baza B-Spline}
They are much more complex. There are two interesting properties that are not part of the Bézier basis functions, namely: (1) the domain is subdivided by knots, and (2) basis functions are not non-zero on the entire interval. In fact, each B-spline basis function is non-zero on a few adjacent subintervals and, as a result, B-spline basis functions are quite "local".



\end{document}