\documentclass[12pt]{article}
\usepackage[T1]{fontenc}
\usepackage[polish]{babel}
\usepackage[utf8]{inputenc}
\usepackage{lmodern}
\usepackage{float}
\selectlanguage{polish}
\usepackage{graphicx}
\usepackage{hyperref}
\title{I ty możesz zostać magistrem}
\date{\today}


\begin{document}	
\maketitle
\tableofcontents

\section{Baza Bernsteina}
Baza Bernsteina ze względu na świetne właściwości numeryczne i geometryczne jest szeroko stosowana w systemach CAD/CAM pomimo faktu, że nie jest trójkątna (traingular ???) i nie jest najszybsza w obliczeniach.\\

Ważne właściwości Bazy Bernsteina:
\begin{itemize}
	\item Formują bazę w $n+1$ wymiarowej przestrzeni $w^{n}$ wszystkich wielomianów stopnia nie większego niż $n$.
	\item Sumują się do 1 dla każdego $t \in R$
	\item Są nieujemne w przedziale $[0,1]$ i dodatnie w $(0,1)$.
	\item Sa symetryczne, tzn. $B^{n}_{i=0...n}(t) = B^{n}_{n-1}(1-t)$
\end{itemize}

\subsection{Algorytm de Casteljau}
Algorytm de Casteljau służy do obliczania wartości wielomianów w Bazie Bernsteina. Jest stabilny numerycznie. 
Niewielkim kosztem możemy uzyskąć nie tylko wartość, ale i pochodną w punkcie. Należy odczytać obie wartości algorytmu dla $n-1$ odjąć je i pomnożyć przez $n$.

\subsection{Twierdzenie Weierstrassa}
Każdą funkcję ciągłą o wartościach rzeczywistych na przedziale domkniętym $[a,b]$ można przybliżyć jednostajnie z dowolną dokładnością wielomianami Bernsteina. Im więcej punktów kontrolnych, tym większa dokładność.

\subsection{Baza Hermite'a}
Jeżeli znamy wartości na krańcach przedziału i znamy wartości pochodnych w tych punktach, to podstawiamy do wzoru i mamy aproksymację funkcji na przedziale.
\subsection{Baza Lagrange'a}
Baza nie jest triangularna (???). Można w niej interpolować.
Wielomiany Lagrange'a pozwalają nam na interpolacje punktów, bez potrzeby rozwiązywania układów współrzędnych.
\subsection{Węzły Czebyszewa}
Interpolacja w węzłach Czebyszewa jest prawie najlepsza (znika efekt Rungego).

\section{Piecewise polynomials}
\subsection{Baza B-Spline}
They are much more complex. There are two interesting properties that are not part of the Bézier basis functions, namely: (1) the domain is subdivided by knots, and (2) basis functions are not non-zero on the entire interval. In fact, each B-spline basis function is non-zero on a few adjacent subintervals and, as a result, B-spline basis functions are quite "local".

\setcounter{section}{22}
\section{Metody interpolacji obrotów}
\begin{enumerate}
	\item LERP
	\item SLERP
	\item Liniowa interpolacja kątów Eulera
	\item SQUAD - interpolacja sekwencji orientacji za pomocą wzoru
\end{enumerate}

\section{Aproksymacja obszaru obrobionego}
Obszar obrobiony możemy aproksymować za pomocą paraboloidy ściśle stycznej. 
$$d(\Delta x, \Delta y) = \frac{1}{2} [\Delta x, \Delta y] \textbf{D}  [\Delta x, \Delta y]^{T}$$
gdzie D to dwuforma (przekształcenie dwuliniowe) określająca przybliżaną powierzchnię.\\
Typowe zadanie z PUSNu: jest dana powierzchnia w postaci implicite. Sprawdź jaki maksymalny promień freza nie spowoduje podcięć lub o ile musimy się przesuwać, żeby frezować z zadaną tolerancją (żeby rowki miały maksymalnie jakąś wysokość).\\
Robimy dwuformę powierzchni i freza i je odejmujemy. Sprawdzamy czy ta dwuforma jest dodatnio określona ($X^{T}DX$ > 0, dla każdego $X$). Jeżeli tak, to nie ma podcięć. Jeżeli nie, to podcięcia mogą wystąpić, ale nie muszą (chyba trzeba sprawdzić kierunki, które chcemy frezować ???).
 
\setcounter{section}{34}
\section{Interpolacja i aproksymacja w bazach B-spline}
\hyperlink{Prezentacja}{http://www.cad.zju.edu.cn/home/zhx/GM/009/00-bsia.pdf} 


\section{Powierzchnie obciętę i standard IGES}
Standard IGES jest standardem międzynarodowym dotyczącym danych topologicznych, geometrycznych i niegeometrzyczne (np. materiały, cechy użytkowe). Na podstawie tego standardu powstał również format pliku o tej samej nazwie pozwalający na zapisanie ponad 150 różnych typów obiektów, np. powierzchni trymowanych.

Powierzchnie obcięte składają się z dwóch części: powierzchni bazowej oraz krzywych trymowania, które wyznaczają obszary trymowania.

\section{Struktury danych reprezentacji B-rep}
%Boundary representation of models are composed of two parts: topology and geometry (surfaces, curves and points). The main topological items are: faces, edges and vertices. A face is a bounded portion of a surface; an edge is a bounded piece of a curve and a vertex lies at a point. Other elements are the shell (a set of connected faces), the loop (a circuit of edges bounding a face) and loop-edge links (also known as winged edge links or half-edges) which are used to create the edge circuits. The edges are like the edges of a table, bounding a surface portion. 
Aby uniknąć częstych obliczeń takich jak np, sprawdzenie czy punkt leży na krzywej warto zapamiętywać informacje przy tworzeniu tych struktur.


Boundary representation składa się z dwóch części: topologii oraz geometrii (powierzchnie, krzywe oraz punkty). Główne elementy topologii to: ściany (faces), krawędzie i wierzchołki. Inne elementy to powłoka (shell) - zbiór połączonych ścianek, pętla - cykl krawędzi ograniczającej ściankę, lopp-edge links (znane także jako skrzydlaczki???) - służą do tworzenia cykli z krawędzi (edge circuits). 

\begin{figure}[H]
	\centering
	\includegraphics[scale=0.5]{Pictures/brep}
	\caption{B-rep od prof. Marciniaka}
\end{figure}

\subsection{Porównanie CSG}
CSG jest drzewem i zapamiętuje kolejne operacje wykonane na prymitywach. Jest budowane z prymitywów i podstawowych operacji boolowskich.\\
B-rep ma większą liczbę operacji, np. wyciągnięcie (extrusion) i wygładzenie krawędzi (chamfer), co pozwala na bardziej "ludzkie" tworzenie modeli.

\section{Metody lokalizacji obliczeń geometrycznych}
Przykładem problemu lokalizacji obliczeń geometrycznych jest wykrycie kolizji dwóch złożonych obiektów. Zamiast sprawdzać każdy ich element z każdym, chcemy uprościć obliczenia, odrzucając obszary, w których wiemy, że kolizja raczej nie zajdzie.

\subsection{Drzewo BSP}
Drzewo BSP (binary space partitioning) - dzielimy przestrzeń na dwie mniejsze dowolną płaszczyzną.\\
Drzewo kd - dzielimy przestrzeń na dwie mniejsze płaszczyzną ortogonalną do osi układu.\\
Octree - dzielimy przestrzeń 3D na osiem sześcianów.\\

\begin{figure}[h!]
	\centering
	\includegraphics[scale=0.5]{Pictures/octree}
	\caption{Przykład quadtree}
\end{figure}

\section{Struktura systemu do projektowania przez podanie ograniczeń}
Zamiast sekwencyjnie rysować scenę (np. najpierw wstawiamy punkt, potem okrąg, który ma w nim środek, następnie styczną itp.), podajemy układ równań, który definiuje nam całą scenę. Jeżeli zmiennych mamy więcej niż równań to część zmiennych musi wprowadzić użytkownik jako dane wejściowe.


\section{Pojęcie naprężenia w materiale. Wektor i tensor naprężenia}
Naprężenie $\sigma$ (ang. stress) to miara gęstości powierzchniowych sił wewnętrznych, występujących w pewnym punkcie przekroju ośrodka ciągłego (ciała wolumetrycznego). Jednostką naprężenia jest paskal. Reprezentowany jako trój-wymiarowy symetryczny tensor drugiego rzędu.

Wektor naprężenia $t$ to gęstość sił działających na element powierzchni w danym punkcie $P \in B$, gdzie $B$ to ośrodek ciągły (ciało). W przypadku sił wewnętrznych wartość $t$ w punkcie $P$ zależy od przekreju. Wektor $t$ jest skierowany wzdłuż wersora normalnego $n$ do powierzchni.

Poprostu: $t$ to wektor z kierunkiem $n$, a $\sigma$ to jego wartość.

\begin{equation}
t_{i} = \sigma_{ij}n_{i}
\end{equation}



\begin{figure}[H]
	\centering
	\includegraphics[scale=0.5]{Pictures/stress.png}
	\caption{}
\end{figure}

\section{Algebra Zewnętrzna}

\subsection{Forma różniczkowa (k-forma)}

\subsection{Pochodna zewnętrzna}

\subsection{Dywergencja}
Dywergencja pola wektorowego to operator różniczkowy przyporządkowujący trójwymiarowemu (dwuwymiarowy też) polu wektorowemu pole skalarne będące formalnym iloczynem skalarnym operatora nabla $\nabla$ z polem. 

Dywergencja to miara ilości strumienia wchodzącego lub wychodzącego z punktu. Dywergencja to tempo ekspansji (ang. expansion, positive divergence) lub skurczania (ang. contraction, negative divergence ) się strumienia.

Jeśli dany punkt zobaczy strumień, który do niego wchodzi to zacznie krzyczeć, że wszystko się zbliża (ang. negative divergence). Jeśli dany punkt zobaczy strumień, który z niego wychodzi to zacznie krzyczeć, że wszystko sie od niego oddala (ang. positive divergence).

\begin{figure}[H]
	\centering
	\includegraphics[scale=1.0]{Pictures/vector_field_div.png}
	\caption{Przykład pola wektorowego pokazujący prędkość strumienia cieczy. Strumień oddala się od początku ukladu współrzędnych (ekspansja). Ta ekspansja (ang. expansion) jest udowodniona przez dodatnią wartość dywergencji div F}
\end{figure}

\subsection{Twierdzenie Stokesa}
Całka formy różniczkowej (k-formy) $\omega$ na brzegu $\partial \Omega$ jakieś rozmaitości (manifold) jest równa całce pochodnej zewnętrznej $d \omega$ na całości $\Omega$

\begin{equation}
\int_{\partial \Omega} \omega = \int_{\Omega} d \omega
\end{equation}

Twierdzenie Gaussa-Ostrogradskiego, podstawowe twierdzenie rachunku całkowego (Fundamental theorem of calculus) i twierdzenie Greena są specjalnymi przypadkami twierdzenia Stokesa.

Tw. Stokesa umożliwia nam zamianę całki powierzchniowej na objętościową (i na odwrót)

\subsection{Twierdzenie Gaussa-Ostrogradskiego}

Znany także jako twierdzenie o dywergencji (Divergence theorem).
Specjalny przypadek tw. Stokesa, w którym funkcja podcałkowa po objętości to dywergencja pola wektorowego $F$

\begin{equation}
\int_{V} (\nabla \cdot F) dV = \int_{\partial V} (F \cdot n) dA
\end{equation}

$F$ to pole wektorowe zdefiniowane w sąsiedztwie $V$. $n$ to pole normalnych (skierowanych na zewnątrz, ang. outward) do brzegu $\partial V$.

Załóżmy, że chcemy napompować koło samochodowe, które jest idealną bryła sztywną (koło nie rozszerzy się po dodaniu powietrza). Co się stanie z powietrzem w środku koła? Powietrzne w środku koło skurczy się.

Jeśli $F$ to pole wektorowe reprezentujące strumień cieczy to dywergencja div $F$ reprezentuje ekspansje lub skurczanie sie tej cieczy. Tw. o dywergencji mówi, że całkowita ekspansja strumienia cieczy w jakims trójwymiarowym regionie $V$ jest równa całkowitemu strumieniowi cieczy na zewnątrz brzegu $V$.

W naszym przykładzie koło stanowi region $V$. Pompowanie powietrza do środka koła w kierunki przeciwnym do normalnej $n$ daje nam ujemna wartość $\int_{\partial V} (F \cdot n) dA$. Co oznacza, że wartość $\int_{V} (\nabla \cdot F) dV$ też jest ujemna, co oznacza, że powietrze skurczyło się. Dokładnie to co założyliśmy.

\end{document}